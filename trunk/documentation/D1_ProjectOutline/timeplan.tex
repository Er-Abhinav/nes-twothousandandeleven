\section{Project time schedule}

We elected eGroupware for our project management matters that prvides:
\begin{itemize}
 \item time management
 \item issue tracking
 \item todo management and
 \item cost tracking
\end{itemize}

\textbf{key explanation for this chapter:} \\
\begin{tabular}{lcl}
Start & ... & Planned starting date of the concerning task\\
End   & ... & Planned finishing date of the concerning task\\ 
resources & ... & Expected resource consumption in percent\\
\end{tabular}


\subsection{Project presentaiton}
\begin{wrapfigure}{r}{0mm}
\begin{tabular}[t]{|lr|}
\hline
Start & kw43\\
End & kw46\\
Resources[\%] & 5\\
\hline
\end{tabular}
\end{wrapfigure}
Project presentation for the first workshop day.\\

\subsection{Documentation}
\begin{wrapfigure}{r}{0mm}
\begin{tabular}[t]{|lr|}
\hline
Start & kw43\\
End & kw3\\
Resources[\%] & 40\\
\hline
\end{tabular}
\end{wrapfigure}
Documentation referes to the entire project processflow and lives therefore over the entire project.\\
One of the main parts in this project is project management and we attach much 
importance to it and thus also on the documentation.\\

Documentation consists of: the documents \cite [NESD1]{NESD1} - \cite [NESD5]{NESD5} 
as well as in code documentation via doxygen documentation language.

\subsection{Dilate testcases}
\begin{wrapfigure}{r}{0mm}
\begin{tabular}[t]{|lr|}
\hline
Start & kw43\\
End & kw3\\
Resources [\%] & 10\\
\hline
\end{tabular}
\end{wrapfigure}
The matter of dialating testcases is to provide apropriate sets of conditions or variables under wich
a tester is able to determine wheter the application is working correctly or not.

Our test cases will include a description of the functionality to be tested and the preparation required
to ensure that the test can be conducted.\\
A test will be, that we will test our system in case of bus load and clock drifts so that we are able to get
significant test result to ensure our clocks can be synchronised in case of a high bus load.

\subsection{Gantt Charts \& Milestones}
\begin{wrapfigure}{r}{0mm}
\begin{tabular}[t]{|lr|}
\hline
Start & kw44\\
End & kw45\\
Resources [\%] & 5\\
\hline
\end{tabular}
\end{wrapfigure}
We are providing a Gantt Chart - that is a task dependent timeline - in form of a image output of our 
project management tool eGroupware. That includes milestones which are points 
in time bound to the timelines in the Gantt Chart.
\subsection{Bus protocol specification}
\begin{wrapfigure}{r}{0mm}
\begin{tabular}[t]{|lr|}
\hline
Start & kw44\\
End & kw45\\
Resources [\%] & 10\\
\hline
\end{tabular}
\end{wrapfigure}
Our bus protocol specification regards layer two and can be found in the documents \cite [NESD2]{NESD2}.

It describes the operational purpose and the qualities of our bus protocol.
\subsection{Testing Implementations}
\begin{tabular}[t]{|lr|}
\hline
Start & kw46\\
End & kw52\\
Resources [\%] & 5\\
\hline
\end{tabular}
\end{wrapfigure}
Over the real process of implementation the hole team has to take care that their 
implementations are tested with the concerning testcase or with a miniature subset of a testcase.\\

Testing each implementation is an important part of the implementation an takes place while
implementing.
\subsection{Hardware Abstraction}
a
\subsubsection{Timer Treiber}
a
\subsubsection{USART Treiber}
a
\subsubsection{LCD Treiber}
a
\subsection{Bus Protokoll Implementation}
a
\subsection{Clocksync Implementation}
a
\subsection{Clocksync Application}
a


