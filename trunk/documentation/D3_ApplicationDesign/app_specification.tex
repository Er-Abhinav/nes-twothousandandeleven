\section{Application specification}

\subsection{Bus load app}
\label{app_specification:bus_load}
Has the benefit of user adjustable provocation load rates for 
further testing

The load app will provide different bus injection methods
\subsubsection{Fault Injection via Buttons}
Is the ability to inject faulty transmissions into the bus by pressing
a button to check the fail-safetyness of the bus and the protocol.

This is meant to be the pendant of an electro magnetic disturbance
injected into the bus cable and should be detected through the
CRC sums by the receiver.

\subsubsection{Overload Simulation via Buttons}
Is the ability to inject heavy bus load by pressing a button to
check the accessibility of the bus if the bus is busy. 
See also: \ref{app_specification:drift_rates} \nameref{app_specification:drift_rates}

\subsection{Drift rates}
\label{app_specification:drift_rates}
The main target of our application is to show drifts of the synchronous 
clocks in our network. 

That means that we are measuring:
\begin{itemize}
 \item The ability of the nodes to synchronize the clocks via the bus 
while it is under load due to the bus load application
 \item The drifts between the clocks before synchronizing
\end{itemize}

\subsection{Visualization of Drift Rates via LCD}
The visualization is planned via a difference bar charts on the 
LCD display of the uC board.

\subsection{Debugging and Monitoring features via PC}
The ability to debug through the PC is one of the first 
feature most developers are looking for. Therefore, it is an
nice to have feature for us to engineer. 

This method could only unfold its full advantage in an emerging 
developer version where the bus will be used in action.
