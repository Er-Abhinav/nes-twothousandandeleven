\section{Requirements}
\label{sec:bus:requirements}

\begin{req}
\label{req:ulftrtp:analyzeable}
\textbf{Analyzeable}: The protocol has to be easy to analyze with respect to worst case timing.
\end{req}

\begin{req}
\label{req:ulftrtp:interfacing}
\textbf{Interfacing}: The protocol design has to strictly follow interface guidelines. This means:
\begin{enumerate}
 \item Lower levels of the protocol can only be accessed by higher levels through the defined layer interfaces.
 \item Higher levels of the protocol cannot be accessed by lower levels. Data to higher layers can only be propagated 
using callback mechanisms. 
\end{enumerate}
Every layer has to be replaceable without any interference with other layers.
\end{req}

\begin{req}
\label{req:ulftrtp:easy migration}
\textbf{Migration}: The protocol has to be migratable in arbitrary applications with minimal effort.
\end{req}

\begin{req}
\label{req:ulftrtp:resource consumption}
\textbf{Resource consumption}: The protocol has to be adaptable to minimal hardware constraints and has to work 
 with small microcontrollers.
\end{req}

\begin{req}
\label{req:ulftrtp:masterless}
\textbf{Masterless}: The protocol has to work without a master.
\end{req}

\begin{req}
\label{req:ulftrtp:length}
\textbf{Bus length}: The protocol has to work over long distances (e.g. 450m)
\end{req}

\begin{req}
\label{req:ulftrtp:baudrate}
\textbf{Baud rate}: The baud rate has to be configureable.
\end{req}

\begin{req}
\label{req:ulftrtp:configuration}
\textbf{Configuration}: The protocol should need minimal configuration effort.
\end{req}

\begin{req}
\label{req:ulftrtp:ft}
\textbf{Fault tolerance}: The protocol should provide basic fault tolerance.
\end{req}

\begin{req}
\label{req:ulftrtp:recovery}
\textbf{Recovery}: The protocol should provide basic recovery from error states.
\end{req}

\begin{req}
\label{req:ulftrtp:clocksync}
\textbf{Clock synchronization}: The protocol should provide basic clock synchronization.
\end{req}
