\section{Layer 2}

Ein Fram setzt sich aus folgendem Inhalt zusammen
\begin{itemize}
  \item Frame-ID
  \item Msg Len
  \item Data
  \item Checksum
\end{itemize}

Frame-ID und Msg-Len bilden den Header jeder Nachricht.
\\
Die Frame-ID setzt sich aus der Message-ID und einer dem Knoten fix zugewiesenen Node-ID zusammen.
Die Aufteilung der Bits zwischen Message- und Node-ID wird nicht spezifiziert und wird je nach Projekt verschieden aufgeteilt.\\
\\
Somit ist es möglich verschiedene Messagetypen von beliebig vielen Nodes zur selben Zeit zu versenden.
Die Priorisierung verschiedenster Messages ist anhand des dominanteren Bits an der höher prioren Stelle der jeweiligen Message-ID gegeben.\\
\\
Im Anschluss an den Header folgen die Nutzdaten deren Länge durch das Feld Msg-Len im Header angegeben wird und an deren Ende die Checksumme, welche sowohl über den Header- als auch den Nutzdatenteil gebildet wird, folgt.\\

\section{Uhrensynchronisation}
Auf den Layer 2 setzt die Uhrensynchronisation / das Messageordering auf.

\section{Benchmarking und Testing}
Diese wird mittels Node X mitprotokolliert und am LCD ausgegeben.
Buttons die sich auf den jeweiligen Nodes befinden (mit diesen verbunden sind) werden dazu verwendet diverse Fehlerszenarien zu aktivieren oder Unschärfen zu verursachen.\\ 